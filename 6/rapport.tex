\documentclass[12pt, a4paper, hidelinks]{article}

% Packages:
\usepackage{graphicx}                   % For figure includes
\usepackage[T1]{fontenc}                % For mixing up \textsc{} with \textbf{}
\usepackage[utf8]{inputenc}             % For scandinavian input characters(æøå)
\usepackage{amsfonts, amsmath, amssymb} % For common mathsymbols and fonts
\usepackage[danish]{babel}              % For danish titles \usepackage{hyperref}                   % For making links and refrences
\usepackage{url}                        % Just because {~_^}
\usepackage{array}                      % ...
\usepackage[usenames, dvipsnames, svgnames, table]{xcolor}
\usepackage{tabularx, colortbl}
\usepackage{verbatim} % For entering code snippets.
\usepackage{fancyvrb} % A "fancy" verbatim (for pseudo code).
\usepackage{listings} % For boxed codesnippets, and file includes. (begin)
\usepackage{lipsum}   % For generating dummy text at this demonstration
\usepackage{float}

% Basic layout:
\setlength{\textwidth}{165mm}
\setlength{\textheight}{240mm}
\setlength{\parindent}{0mm}
\setlength{\parskip}{\parsep}
\setlength{\headheight}{0mm}
\setlength{\headsep}{0mm}
\setlength{\hoffset}{-2.5mm}
\setlength{\voffset}{0mm}
\setlength{\footskip}{15mm}
\setlength{\oddsidemargin}{0mm}
\setlength{\topmargin}{0mm}
\setlength{\evensidemargin}{0mm}


\newcolumntype{C}[1]{>{\centering\arraybackslash}p{#1}}

% Colors:
\definecolor{KU-red}{RGB}{144, 26, 30}

% Text Coloring:
\newcommand{\green}[1]{\textbf{\color{green}{#1}}}
\newcommand{\blue} [1]{\textbf{\color{blue} {#1}}}
\newcommand{\red}  [1]{\textbf{\color{red}  {#1}}}


% ************************* Start Document *****************
\begin{document}

% ************************* Page Header ********************
\begin{minipage}[b]{1.0\linewidth}
\includegraphics[height=30mm]{bilag/KULogo}

\vspace*{-16ex}
\begin{center}
    {\Large \bf DMA} \vspace*{1ex} \\
    {\large Ugeopgave 6} \vspace*{1ex} \\
    {\large Matti Andreas Nielsen  } \\
    {\large \today{}  }
\end{center}
\vspace*{-3pt}
{\color{KU-red}\hrule}
\end{minipage}
\vspace{2ex}

% **************** Assignment Starts Here ******************
\tableofcontents \newpage

\section{Del}
$$ F_{0} = 0, F_{1} = 1, F_{n} = F_{n-1} + F_{n-2}  $$

\subsection{}

Basis step: 
$$ F_{0} + F_{1} = 1 $$
$$ 2^1 = 2 $$
$$ 1 \leq 2   $$

Induktions hypotese: $$ F_{n} \leq 2^n $$ \\

Så vores base case holder\\

Induktionstrin: \\
Vi antager at $$ F_{n}(1) $$ er sandt, nu vil vi så vise at det også gælder for $$ F_{n}(n+1) $$ \\

$$ F_{n+1} = F_{n} + F_{n-1} $$
$$ 2^{n+1} = 2*2^n = 2^n+2^n  $$

Fordi vores induktions hypotese siger at $$ F_{n} \leq 2^n $$ 
og fordi at $$ F_{n-1} \leq F_{n} $$ kan vi
opstille det følgende udtryk

$$ f_{n} + f_{n-1} \leq 2^{n} + 2^{n} $$

\subsection{}
Vi vil gerne vise at det følgende er sandt
$$ F_{n} \geq \frac{3}{2}^{n-1} \text{,} \forall n \in \langle 6, 7, 8, \ldots \rangle    $$
ved hjælp af induktion.\\

Basis step 
$$ F_{6} = 8 $$
$$ \frac{3}{2}^{6-1} = 7.59 $$
$$ 8 \geq 7.59 $$

Vi regner en til sådan at vi kan bruge stærk induktion
$$ F_{7} = 21 $$
$$ \frac{3}{2}^{7-1} = 16 $$
$$ 21 \geq 16 $$

grunden til at jeg er interesseret i at bruge stærk induktion, er at jeg så vil kunne antage at 
$$ \forall n \in \langle 6, 7, 8, \ldots , n \rangle F_{n} \geq \frac{3}{2}^{n-1} \rightarrow F_{n+1} \geq \frac{3}{2}^{n}    $$
i modsætning til svag induktion hvor jeg kun ville kunne antage at 
$$ \exists{n} \in \langle 6,7,8, \ldots , n \rangle F_{n} \geq \frac{3}{2}^{n-1} \rightarrow F_{n+1} \geq \frac{3}{2}^{n}  $$ 

Induktionshypotese eller induktionsantagelse er nu at   $$ F_{n} \geq \frac{3}{2}^{n-1} $$ og fordi at vi bruger stærk induktion
kan vi også sige at $$ F_{n-1} \geq \frac{3}{2}^{n-2} $$ 

Induktionstrin\\
Det vi så gerne vil vide er om det også gælder for $$ n+1 $$ så
vi indsætter dette i stedet for $$ n $$ så vores udtryk kommer til at se således ud
$$ F_{n+1} = F_{n} + F_{n-1} $$
$$ \frac{3}{2}^{(n+1)-1} = \frac{3}{2}^{n} $$

Det vi så gerne vil bevise er at følgende udtryk 
$$ F_{n} + F_{n-1} \geq \frac{3}{2}^{n} $$ 
stadig væk holder.

Ideen her, er så at finde et udtryk, som vi kan sætte imellem disse 2 udtryk. 
Her kan vi gøre brug af at når man plusser to større størrelser sammen vil det stadig være 
større end når man plusser to mindre størrelser sammen. Derfor kan vi sige at
$$ F_{n} + F_{n-1} \geq \frac{3}{2}^{n-1} + \frac{3}{2}^{n-2} $$ vores mål er nu at placere vores 
højre udtryk således $$  \frac{3}{2}^{n-1} + \frac{3}{2}^{n-2} \geq  \frac{3}{2}^{n} $$

udtrykket
$$  \frac{3}{2}^{n-1} + \frac{3}{2}^{n-2} $$ 
kan man via sine potensregneregler omskrive til

$$ \frac{3}{2}^{n} \frac{3}{2}^{-1} +  \frac{3}{2}^{n} \frac{3}{2}^{-2} $$

Vi kan så trække $$ \frac{3}{2}^{n} $$ ud og får

$$ \frac{3}{2}^{n} ( \frac{3}{2}^{-1} + \frac{3}{2}^{-2} ) $$

som giver $$ \frac{3}{2}^{n} \frac{10}{9} $$

Og her er det klart at når vi ganger en størrelse med et tal over 1, så vil det altid være større end
det originale tal, så vi har derfor vist at

$$ \frac{3}{2}^{n} \frac{10}{9} \geq \frac{3}{2}^{n} $$

og at

$$ F_{n} + F_{n-1} \geq \frac{3}{2}^{n-1} + \frac{3}{2}^{n-2} \geq \frac{3}{2}^{n} $$

som koget ned kan skrives som præcis det vi gerne ville bevise med vores induktionstrin, nemlig at

$$ F_{n} + F_{n-1} \geq \frac{3}{2}^{n} $$

og dermed har vi også bevist at 
$$  \forall n \in \langle 6, 7, 8, \ldots \rangle F_{n} \geq \frac{3}{2}^{n-1}  $$

\subsection{}

$$ \log{F_{n}} \leq \log{2^n} $$
$$ \log{2^{n}} = n * \log{2} $$
$$ n \leq \log{F_{n}} \leq n $$

$$ \log{F_{n}} \leq \log{\frac{3}{2}^{n-1}} $$

$$ (n-1) * \log{(\frac{3}{2})} \leq \log{F_{n}} \leq n * \log{2} $$

jeg ved her at jeg skal bruge reglen S6 til at fjerne konstanterne og på en eller anden måde 
skal jeg have fjernet -1 fra mit n, men jeg har ikke kunne lave opgaven helt færdig.

\section{}
\subsection{}
Mul metoden laver heltals division af a og b, som returnere true hvis b går op i a, false hvis ikke og variablen y tæller hvor mange gange b går op i a.
Desuden indeholder x variablen rest værdien af divisionen.

Hvis vi kalder $$ Mul(6, 3) $$ vil når while loopet terminere returnere true, fordi x vil være 0, udover det er y=2  fordi 3 går op i 6 2 gange. \\
$$ Mul(7, 3) $$ når while loopet terminere returnere false, fordi x vil være -1 hvor hvis man vender fortegnet om har man resten af divisionen, udover det vil y=3.\\
$$ Mul(3, 7) $$ vil terminere med det samme og returnere false, så det viser at man ikke bare ville kunne returnere x som en rest, i det her tilfælde burde den jo være -4 men fordi vi aldrig hopper ind i løkken er x bare 3. \\
$$ Mul(0, 7) $$ ville terminere med true.\\ 

\subsection{}
I $$ Mul(a, b) $$ er vores basis step at while løkken kører 0 gange, 
for at få den til at kører 0 gange kan vi eventuelt håndkører algoritmen når vi eksekverer 
algortimen med $$ Mul(1, 3) $$
første gennemløb $$ x = 1, y = 0 $$ og hvis vi indsætter dette i vores funktion
$$ x_{n} + by_{n} = a $$ får vi $$ 1 + 3 * 0 = 1 $$ som vi kan se passer det med at $$ a = 1 $$
men en anden ting vi kan gøre som måske giver mere mening er at blot regne det med bogstaver 
$$ x_{n} + by_{n} = a $$
$$ x_{n} +by_{n} = x_{n} + by_{n} $$
Første gennemløb, hvor $$ a = x_{n}, y_{n} =  0 $$ giver os
$$ x_{n} + by_{n} = a + b * 0  $$
$$ x_{n} + by_{n} = a $$
dermed er vores basis case accepteret \\

Induktionskridt \\
Her kan vi ikke bare udskifte $$ n $$ med $$ n+1 $$ som vi er vant til, siden det kun 
viser indexet i vores algoritme. Her skal vi ind og kigge på hvordan vores $$ x_{n}, y_{n}  $$ bliver ændret inde i vores 
while løkke når vi kører en til iteration. Her kan vi så aflæse at
$$ x_{n+1} = x_{n} - b $$
$$ y_{n+1} = y_{n} + 1 $$

Vi indsætter så $$ n+1 $$ i det oprindelige udtryk
$$ x_{n+1} + by_{n+1} = a $$
Vi udskifter udtrykene med dem ovenfor og får
$$ x_{n} - b + b(y_{n} + 1) = a $$
Vi ganger paranthesen ud
$$ x_{n} - b + by_{n} + b = a $$
De to b'er går ud med hinanden og vi får det oprindelige udtryk 
$$ x_{n} + by_{n} = a $$

\subsection{}
Det første jeg gerne vil vise er at vores while løkke rent faktisk terminere, det er noget som skal være sandt 
for at nogle af de andre muligheder kan gå i opfyldelse. 
Vi får opstillet et krav til vores variabler, som er følgende:
$$ a,b \in \mathbb{Z^+}  $$ vores while løkkes condition(tilstand?) består af $$ x >= b $$ og fordi at 
både a og b er positive heltal samtidig med at vi for hver iteration trækker b fra x gælder det at  $$ \lim_{x \to 0} $$ imens b forbliver 
den samme værdi, og vores while løkke vil dermed altid terminere.
Man kan dele resten af vores cases op i 2 muligheder, vi har $$ a < b $$ som altid vil returnere false, fordi $$ b|a $$ aldrig vil være muligt.
Den sidste case vi har er $$ a \geq b $$ som kan enten returnere true eller false, afhængigt af om $$ b|a $$ hvis dette er sandt, returnere vi TRUE, og hvis det er falskt 
returnere vi false. Grunden til dette er at vi minusser b med a for hver eneste iteration, det betyder at hvis b ikke går op i a vil x aldrig blive 0, og dermed vil vi ikke returne true. 




\end{document}
